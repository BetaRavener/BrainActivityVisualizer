%=========================================================================
% (c) Ivan Ševčík, 2015
abb:
EEG
PC

\chapter{Introduction}
% Uvod do temy
The brain is central organ of human nervous system (cite?) and as such has very
important role in almost every activity. However, its complexity makes it
difficult to study and understand. Rapid development of computers in recent
decades provided partial solution to this as it allowed for mapping and
monitoring both structure and activity of brain with high precision,
undoubtely resulting in great increase in the rate of new discoveries.

% Aktualny stav, problemy
But raising interest in EEG technology and brain-computer interfaces also means
that there is an evident need for user interfaces and applications capable of
processing these signals for various purposes. One such purpose is visualization
that allows researchers and users better comprehend measured data as these are
usually just binary values that may be represented as integer or decimal
numbers, therefore hard for humans to interpret. (cite?) Another issue is that
generally users are not interested in raw values, but in some features that
signal carries such as intensity at certain frequency or specific patterns that
represent certain activity performed by measured subject.

% Ciele
Our goal is therefore to create application implementing various means of signal
processing and visualization, including classification by frequency, graphs
showing waveform of measured signal in time domain and both two and
three-dimensional models displaying brain acitivity. Another important task will
be to create intuitive and clean, yet highly customizable graphical user
interface that will be responsive also for real-time signal inputs. The
application should, however, be able to process also large offline data measured
earlier without noticeable performance decrease.

% O com to tu je
In this work we will first present background information about brain and
methods for measuring its activity, particularly EEG, discuss the processing of
signal on computer and methods for visual representation. Finally, we will
present our own application in detail, starting with concept and moving on to
implementation.
 
\chapter{Human brain}
The necessary information for biological background will be
provided here. The human brain is central unit of his nervous system..
\section{Brain biology}
\emph{2 pages}
Presenting relevant imformation about brain, its parts and their functions..
Electrical signals..  (nieco k tomu co tie el. impulzy robia.. komunikacia etc.)
Discusing in short various means to monitor brain and its activity -- EEG, 
CT scans, optical infrared spectroscopy. Showing images of brain.
Mostly citations to medicine research.

\section{Electroencephalography}
\emph{4 pages}
The nature of human nervous system is electrical. It has been proved that the
variation of surface electrical potential reflects activities performed in
underlying brain structures. EEG is a method for measuring these potential
variantion using an array of electrodes placed on measured subject's scalp and
recording them for later processing.\cite{eegClass}

EEG is commonly used for measurements in medical and research areas. Its main
advantage over other methods is speed as it can record respond to stimulus
within fractions of a second.\cite{eegFund}
% Applications
Applications include:
\begin{itemize}
  \item Research -- monitoring during cognitive and motoric tasks
  \item Medical -- diagnosis of brain diseases
  \item Human computer interaction -- performing commands using brain activity  
\end{itemize}

\subsection{Measuring equipment}
The setup used for encephalographic measurements usually consist of: 
\begin{itemize}
  \item electrodes with conductive media
  \item amplifiers with filters
  \item A/D converter
  \item recording device
\end{itemize}
% Electrodes
The electrodes exists in various forms but in this work we will focus on
headbands and electrode caps as these are non-invasive and require the least
expertise. These caps usually consists of small discs serving as electrodes made
of very conductive material such as gold, silver or stainless steel. Commonly
used electrodes are made of Ag-AgCl disks with diameter of \SIrange[range-units
= single]{1}{3}{\mm} and have long leads that can be plugged into an
amplifier. \cite{eegFund} The discs can be in direct contact with scalp and
measure electric potential produced by brain, but conductive media in form of a
gel or paste is often used to increase conductivity even more by lowering
contact impedance and improve readings at the lowest level, while it also helps
electrodes to stick to surface.

% Cap
The electrodes are pre-mounted on a silicon cap in locations according to
standardized placement systems that will be discussed later in this section.
This both speeds up the setup stage and allows for unified mapping of electrodes
to head surface, but fails to account for different shape and features of each
individual's head. \cite{eegFund} This problem is usually solved by involving a
method for locating 3D coordinates of electrode positions on the head. Such
methods include using magnetic field digitizer or elaborate algorithms that can
calculate position of each electrode from few distance and angle measurements by
utilizing specific properties of standard placement system. \cite{rapidPos}

% Amplifiers
The strength of signal measured by electrodes is in order of microvolts and
normally range from \SIrange{10}{500}{\uV}.\cite{neuralAmp} An amplifier is
therefore needed to bring the amplitude to higher levels so it can be processed
by electrical circuits and components. The requirements on amplifiers for use
with EEG are high as they have to provide amplification only for physiological
signal which should not be distorted in any way, reject superimposed noise and
interference signals, and protect both the equipment and measured subject from
current surges. Amplifiers conforming to these are known as biopotential
amplifiers. \cite{biopotAmp}
% Filters
A bandpass filter is then used to limit frequencies into a ceratin range of
interest. In some cases it may be already a part of an amplifier.
Another common filter is notch filter which is used to filter out noise
at frequency of power line. Depending on country it may be set to
\SIlist[list-units = single, list-pair-separator = { or }]{50;60}{\Hz}.
\cite{deltaCompNREM} Such filter is only used if it is desirable to keep also
high frequencies as the information of interest usually lies bellow this
frequency.
% Trend in amplifier miniaturization
The trend in development of neural recording devices is heading in direction of
fully implanted systems, therefore energy efficient amplifiers are being
designed lately that are very small in size, may run on battery for long period
of time and disipate only little heat so they don't damage surrounding tissue.
\cite{neuralAmp}

% A/D converter
An A/D converter unit is then used to convert analog values to digital
representation. The converter should have resolution of at least 12 bits.
With high number of electrodes, analog multiplexers are sometimes used to lower
the number of necessary converter units at the expense of limited update
frequency. The sampling frequency should be at least double of the highest
recorded frequency, for example the upper frequency limit set for bandpass
filter. This is known as Nyquist-Shannon sampling theorem and is fundamental for
correct signal reconstruction without aliasing artifact. The preffered sampling
frequency is \SIrange{256}{400}{\Hz}. \cite{guidDigEEG}

% Recording device
Converted samples from A/D converters are then stored in memory for further
processing which will be presented later in this section. A recording device
might be represented by a computer or a different kind of equipment.
%TODO: mozno nieco viac?

\subsection{Electrode placement systems}
%TODO: Zmienka o vybere referencnej elektrody
% Uvod
The need for standardized electrode placement was evident as early as 1947 when
the first Internation EEG congress held place in London. Various method of
standardization were proposed and this effort finally resulted in the definition
of 10-20 electrode system in 1958 by H.H. Jasper. \cite{elSys} 
% Popis 10-20
This system consists of 21 electrodes placed evenly between certain landmarks.
(dokoncit)

% Popis 10-10
The 10-20 system offers only limited resolution but it was sufficient at the
time of its creation because the main limiting factor was technology. However,
with technological advances in computers and signal processing, using more than
21 channels became feasible and naturally many seeked to take advantage of
higher resolution. With more electrodes than what is 10-20 model capable of
accommodating, first extension to standard placement system was needed. The
extension was proposed in 1985 and extended the number of electrodes to 74. It
uses additional coronal contours lying halfway between original contours and
combines the labels of these original contours to create new label.
For example, the contour lying betweeen original countours F and C will be
labelled FC. \cite{elSys}

% Popis 10-5
The 10-5 electrode placement systems further extends the 10-10 system to allow
even higher number of channels to be used, as systems with over 128 channels are
not uncommon any more and have became commercially available. The idea in
placement of new electrodes is simmiliar to how 10-10 system extends from
10-20, creating coronal contours halfway between the original ones. The
nomenclature uses naming in similiar fashion as geographical directions. For
example, direction between North and North-West is labelled as North-North-West
and so contour between C-contour and CP-contour will be labelled CCP-contour.
This doubles the number of contours and also the number of electrodes in
contour, increasing the number of electrodes approximately 4 times. The total
number of possible locations is arround 345 and may vary depending on how many
electrodes are included for the most inferior rows. \cite{elSys} This system is
in proposition stage and is yet to be recognized as standard but it is currently
only work in this area, therefore can be considered as relevant and some
equipment already adheres to it.

\subsection{Measurment procedure}
Before measurment, the electodes are cleaned thoroughly to remove any impurities
that could have an impact on it. The skin is also cleaned from oils and brushed
from dried parts. It is important to adhere to strict hygiene as this could
cause irritation and inflamation and with repeated measurements may develop into
an infection.\cite{eegFund} The subject is then seated comfortably to eliminate
any unnecessary movement, because it could spoil the measured data. (cite?)
Elecetrodes are placed on subject's scalp and recording can begin. Subject is
usually instructed to perform specific tasks such as imagining moving certain
part of body, rotating an object in 3D or focusing on specific part of display.
\cite{bcComm} For other purposes, subjects may be monitored while performing
ordinary activities or during sleep.

\subsection{Data analysis}
The recorded data are subject to further processing using numerical filters and
transforms to improve signal-noise ration and extract desired features. 
% Clasification by frequency band
One such feature is brain activity at certain frequency. The frequency of brain
waves range from \SIrange{1}{150}{\Hz}. This range can be further divided
into six common categories\cite{dominantF}:
\begin{itemize}
  \item Delta waves -- below 4 Hz
  \item Theta waves -- between 4 and 8 Hz
  \item Alpha / Mu waves -- between 8 and 13 Hz
  \item Beta waves -- between 13 and 30 Hz
  \item Gamma waves -- between 30 and 80 Hz
  \item High gamma waves -- above 80 Hz
\end{itemize}
A healthy person exhibits brain waves mostly in the first four categories. Delta
waves are normally distributed over scalp and may be significant during deep
sleep, in childhood or in serious organic brain disease.\cite{eegClass} Theta
waves are related to cognitive tasks such as working memor and error monitoring.
Alpha power is increased during inattention and lack of visual input, therefore
related to perception. Mu power, on the other hand, is linked to movement and is
decreased when person performs a motor action. The distinction between alpha and
mu waves is made by considering location of electrode. While alpha waves form at
the back of the scalp, mu waves can be observed in a strip of brain from ear to
ear. Finally beta waves are low amplitude, high frequency waves that have power
reduced at the onset of movement, rebound if the movement is sustained and are
enhanced if the movement is suppressed. Each band is therefore related to
certain group of activity in brain and can be used for diagnosis as abnormal
resting powers in certain bands were linked to various mental and physiological
diseases.\cite{dominantF}

% Classifier reference
Many researchers proposed implementation of classifiers capable of identifying
these frequencies in signal using various approaches. One such method that will
be used in this work is described in detail in section \ref{sec:sigProc}.

% Classification by pattern
Another set of features are patterns that represent certain activity more
clearly than classification by frequency. For example, the imagined movement of
finger can be distinguished from imagined movement of leg with this method. The
classification accuracy is, however, highly individual and the classification
system usually needs tuning or calibration process for each user.\cite{bcComm}
Common approach to this problem is training a neural network that can classify
patterns in brain waves with accuracy high enough for normal usage.

\chapter{Computer assistance}
\section{Brain-Computer interface} 
\section{Signal processing}
\label{sec:sigProc}
\emph{2 pages}
Introduction to sampling, fourier transform, various signal filters..
\subsection{EEG waves classifier}
In this work we will use classifier using Wavelet transform and Fourier
transform as proposed by M.M. Shaker. \cite{eegWaveFt} 
\section{Data visualization methods}
\emph{4 pages}
\subsection{2D representations}
graph, summary overview table, histogram, unfolded brain map..
\subsection{3D models}
mesh, volumetric/voxel graphics.. 
Means to obtain three-dimensional brain model - CT, modelling..
\section{Existing solutions}
\emph{2-3 solutions, 1 page each}
Analysis of existing solutions, what could serve as inspiration, what could be
improved from all points of view (ease of installation, usability, features
friendliness, signal processing, settings, extensibility..)
\subsection{A}
\subsection{B}
\subsection{C}
\chapter{Concept and design}
Theoretical concept for practical part.
\section{User interface}
\emph{2 pages}
Drawings, mockups, layouts, command prompt parameters.
\section{Code structure}
\emph{3 pages}
\subsection{Logical division}
The division on subsystems for graphics context setup, signal processing, gpu
rendering procedures, user interface and model importer.
\subsection{Classes and communication}
Structure of code defined by C++ classes, probably should be just global
overview with references to doxygen documentation, maybe some graphical code
maps.
\chapter{Implementation}
\emph{? pages}
Will cover implementation details, algorithms, significant mechanisms, shader
programming.. The logical division defined earlier will be probably used to
describe each part separately.
\section{Tools and libraries}
\emph{0.5-1 page}
Short enumeration, links or references to websites.
\section{Input data processor}
\subsection{Input format}
\subsection{Filters}
\section{Model importer}
\section{3D rendering system}
\section{Plugin system}
\section{Graphical user interface}
\section{\ldots}
\chapter{Results}
Summarization of results obtained by this work, final graphs and tables,
higlighting of the most interesting parts, presentation of user 
research -- opinions, suggestions,
feture requests..
\chapter{Discussion}
Discussing the results in regard to referenced literature and their results.
Showing significance of findings, questioning them, providing several
perspectives and means for argumentation.
\chapter{Conclusions}
Refering to introduction and checking with goals, if everything required was
done, what couldn't be done and why. Placing the work into wider context of
relevant areas of research -- medicine, bioinformatics, user interfaces..
Part about possible plans for future, improvment, enhancements, folow up..
%=========================================================================
