%============================================================================
% tento soubor pouzijte jako zaklad
% (c) 2008 Michal Bidlo
% E-mail: bidlom AT fit vutbr cz
%============================================================================
% kodovaní: iso-8859-2 (zmena prikazem iconv, recode nebo cstocs)
%----------------------------------------------------------------------------
% zpracování: make, make pdf, make desky, make clean
% připomínky posílejte na e-mail: bidlom AT fit.vutbr.cz
% vim: set syntax=tex encoding=latin2:
%============================================================================
\documentclass[english,cover]{fitthesis} % odevzdani do wisu - odkazy, na ktere
% se da klikat \documentclass[cover,print]{fitthesis} % pro tisk - na odkazy se neda klikat
%\documentclass[english,print]{fitthesis} % pro tisk - na odkazy se neda klikat
%      \documentclass[english]{fitthesis}
% * Je-li prace psana v anglickem jazyce, je zapotrebi u tridy pouzit 
%   parametr english nasledovne:
%\documentclass[english]{fitthesis}
% * Neprejete-li si vysazet na prvni strane dokumentu desky, zruste 
%   parametr cover

% zde zvolime kodovani, ve kterem je napsan text prace
% "latin2" pro iso8859-2 nebo "cp1250" pro windows-1250, "utf8" pro "utf-8"
%\usepackage{ucs}
\usepackage[utf8]{inputenc}
\usepackage[T1, IL2]{fontenc}
\usepackage{url}
\usepackage{siunitx}
\usepackage{amsmath}
\usepackage{listings}
\usepackage[acronym,nonumberlist,nomain]{glossaries}
\usepackage[section]{placeins}
\DeclareUrlCommand\url{\def\UrlLeft{<}\def\UrlRight{>} \urlstyle{tt}}

%zde muzeme vlozit vlastni balicky


% =======================================================================
% balíček "hyperref" vytváří klikací odkazy v pdf, pokud tedy použijeme pdflatex
% problém je, že balíček hyperref musí být uveden jako poslední, takže nemůže
% být v šabloně
\ifWis
\ifx\pdfoutput\undefined % nejedeme pod pdflatexem
\else
  \usepackage{color}
  \usepackage[unicode,colorlinks,hyperindex,plainpages=false,pdftex]{hyperref}
  \definecolor{links}{rgb}{0.4,0.5,0}
  \definecolor{anchors}{rgb}{1,0,0}
  \def\AnchorColor{anchors}
  \def\LinkColor{links}
  \def\pdfBorderAttrs{/Border [0 0 0] }  % bez okrajů kolem odkazů
  \pdfcompresslevel=9
\fi
\fi

%Informace o praci/projektu
%---------------------------------------------------------------------------
\projectinfo{
  %Prace
  project=BP,            %typ prace BP/SP/DP/DR
  year=2015,             %rok
  date=\today,           %datum odevzdani
  %Nazev prace
  title.cs={Vizualizace elektrických signálů mozku},  %nazev prace v cestine
  title.en={Brain Signals Visualization}, %nazev prace v anglictine
  %Autor
  author={Ivan Ševčík},   %jmeno prijmeni autora
  %author.title.p=Bc., %titul pred jmenem (nepovinne)
  %author.title.a=PhD, %titul za jmenem (nepovinne)
  %Ustav
  department=UITS, % doplnte prislusnou zkratku: UPSY/UIFS/UITS/UPGM
  %Skolitel
  supervisor= Michal Košík, %jmeno prijmeni skolitele
  supervisor.title.p=Ing.,   %titul pred jmenem (nepovinne)
  %supervisor.title.a={Ph.D.},    %titul za jmenem (nepovinne)
  %Klicova slova, abstrakty, prohlaseni a podekovani je mozne definovat 
  %bud pomoci nasledujicich parametru nebo pomoci vyhrazenych maker (viz dale)
  %===========================================================================
  %Klicova slova
  keywords.cs={Klíčová slova v českém jazyce.}, %klicova slova v ceskem jazyce
  keywords.en={Klíčová slova v anglickém jazyce.}, %klicova slova v anglickem jazyce
  %Abstract
  abstract.cs={Výtah (abstrakt) práce v českém jazyce.}, % abstrakt v ceskem jazyce
  abstract.en={Výtah (abstrakt) práce v anglickém jazyce.}, % abstrakt v anglickem jazyce
  %Prohlaseni
  declaration={Prohlašuji, že jsem tuto bakalářskou práci vypracoval samostatně pod vedením pana ...},
  %Podekovani (nepovinne)
  acknowledgment={Zde je možné uvést poděkování vedoucímu práce a těm, kteří poskytli odbornou pomoc.} % nepovinne
}

%Abstrakt (cesky, anglicky)
\abstract[cs]{
Tato práce pojednáva o problematice zpracování a vizualizace mozkových signálů
a jejich využití pri analýze dat měřených pomocí EEG. 
Práce obsahuje úvod do anatomie mozku a povahy mozkové aktivity, který je pak
využit při popisu jedné z metod meření této aktivity, konkrétně EEG. Dále se práce
zabýva metodami zpracování signálů, především filtrací, a metodami vizualizace.
Rovněž je zde uveden prěhled předchozích rešení zabývajících se
vizualizací biosignálů, které sloužily jako výchozí představa při tvorbě konceptu 
vlastní aplikace.
Tato aplikace byla napsána v jazyce C++ s využitím knihovny Qt pro vytvoření 
multiplatformního grafického uživatelského rozhraní. Důležitým aspektem bylo
také využití rozhraní OpenGL, které umožnilo hardwarově akcelerovat vykreslování
a tím pádem dosáhnout uspokojivé překreslovací frekvence. Dále bylo využito několik
knihoven, ktére implementují načtení EEG dat ze souboru, rychlou Fourierovou transformaci,
matematické operace v 2D a 3D prostoru, a načítání 3D objektů ze souboru.
Implementace pak řeší několik problémů. V první řade je to vykreslování
2D a 3D modelů mozku a vizualizace mozkové aktivity nad těmito modely formou animovaných
elektrod. V důsledku
využití OpenGL bylo nutné pro tyto operace napsat příslušné shader programy, které
řídí grafickou kartu v procesu vykreslování. Aplikace obsahuje také prohlížeč 
mozkové aktivity ve formě vícenásobného grafu. U něj bylo nutné řešit vykreslování
velkého množství dat, které je počas EEG sezení naměřeno. To je řešeno decimací
signálu metodou, při které se z několika vzorků vybere maximum a minimum, která
jsou následně vykreslena. Tato metoda poskytuje výrazné zrychlení běhu aplikace
při zachování charakteristické křivky signálu. 
Protože EEG data sou ve své podstatě signály, bylo žádoucí poskytnout také metody
pro zpracování signálů přímo v aplikaci. Aplikace tedy podporuje filtraci signálů
pomocí dolní a horní propusti a dvojici váhovacích oken -- Hammingovo a Blackmanovo.
Implementace této části byla testována porovnáním výsledků různých konfigurací s
výstupem programu MATLAB, ve kterém byli vytvořeny shodné filtry.
Výsledná aplikace představuje nástroj pro analýzu mozkových signálů kombinující
prohlížení signálů s prostorovými vizualizacemi. Prostorová vizualizace poskytuje
přehledný způsob pro zkoumání průběhu všech signálů v čase a umožňuje snadnou 
vizuální identifikaci mozkové aktivity. Ta může být podrobněji analyzována právě pomocí
prohlížeče signálů. 
Aplikace zatím nepodporuje řadu žádaných funkcí, např. nahrávaní a zpracování signálů
v reálnem čase. Tyto nedostatky spolu s navrhovanými řešeními jsou uvedena v závěru práce.
Nicméně, součásna implementace poskytuje vhodný základ pro plánovaný budoucí vývoj.
}
\abstract[en]{
EEG is a non-invasive method for measuring brain activity that gives an important insight into mental processes. This work presents an application that serves as a visual EEG data analysis tool. The application concept was based on the existing work done in this field. The application combines signal display in form of a view composed of multiple charts with 2D and 3D visualizations that are helpful in recognition of particular brain activity. In addition, it implements convenient signal processing methods. The thesis is concluded by a summarization of accomplishments and future work's directions.
}

%Klicova slova (cesky, anglicky)
%\keywords[cs]{Sem budou zapsána jednotlivá klíčová slova v českém jazyce, oddělená čárkami.}
%\keywords[en]{Sem budou zapsána jednotlivá klíčová slova v anglickém jazyce, oddělená čárkami.}

%Prohlaseni
\declaration{Prohlašuji, že jsem tuto bakalářskou práci vypracoval samostatně pod vedením pana Ing. Michala Košíka. Uvedl jsem všechny literární prameny a publikace, ze kterých jsem čerpal.}

%Podekovani (nepovinne)
\acknowledgment{Týmto by som sa rád poďakoval svojmu vedúcemu Ing. Michalovi Košíkovi za odbornú pomoc a cenné rady pri vypracovaní bakalárskej práce.}

\makeglossaries

\begin{document}
  % Vysazeni titulnich stran
  % ----------------------------------------------
  \pagenumbering{Alph}
  \maketitle
  \pagenumbering{arabic}
  \setcounter{tocdepth}{1}
  % Obsah
  % ----------------------------------------------
  \tableofcontents
  
  % Seznam obrazku a tabulek (pokud prace obsahuje velke mnozstvi obrazku, tak se to hodi)
  % \listoffigures
  % \listoftables 

  % Text prace
  % ----------------------------------------------
  %=========================================================================
% (c) Ivan Ševčík, 2015
abb:
EEG
PC

\chapter{Introduction}
% Uvod do temy
The brain is central organ of human nervous system (cite?) and as such has very
important role in almost every activity. However, its complexity makes it
difficult to study and understand. Rapid development of computers in recent
decades provided partial solution to this as it allowed for mapping and
monitoring both structure and activity of brain with high precision,
undoubtely resulting in great increase in the rate of new discoveries.

% Aktualny stav, problemy
But raising interest in EEG technology and brain-computer interfaces also means
that there is an evident need for user interfaces and applications capable of
processing these signals for various purposes. One such purpose is visualization
that allows researchers and users better comprehend measured data as these are
usually just binary values that may be represented as integer or decimal
numbers, therefore hard for humans to interpret. (cite?) Another issue is that
generally users are not interested in raw values, but in some features that
signal carries such as intensity at certain frequency or specific patterns that
represent certain activity performed by measured subject.

% Ciele
Our goal is therefore to create application implementing various means of signal
processing and visualization, including classification by frequency, graphs
showing waveform of measured signal in time domain and both two and
three-dimensional models displaying brain acitivity. Another important task will
be to create intuitive and clean, yet highly customizable graphical user
interface that will be responsive also for real-time signal inputs. The
application should, however, be able to process also large offline data measured
earlier without noticeable performance decrease.

% O com to tu je
In this work we will first present background information about brain and
methods for measuring its activity, particularly EEG, discuss the processing of
signal on computer and methods for visual representation. Finally, we will
present our own application in detail, starting with concept and moving on to
implementation.
 
\chapter{Human brain}
The necessary information for biological background will be
provided here. The human brain is central unit of his nervous system..
\section{Brain biology}
\emph{2 pages}
Presenting relevant imformation about brain, its parts and their functions..
Discusing in short various means to monitor brain and its activity -- EEG, 
CT scans, optical infrared spectroscopy. Showing images of brain.
Mostly citations to medicine research.



\section{Electroencephalography}
\emph{4 pages}
Providing broad and exhaustive information about this method for monitoring as
this is basis for all following work. 
Again a lot of citations here.


\subsection{The setup -- need change}
The setup used for encephalographic measurments usually consist of: 
\begin{itemize}
  \item electrodes with conductive media
  \item amplifiers with filters
  \item A/D converter
  \item recording device
\end{itemize}

% Electrodes
The electrodes also exists in various forms but in this work we will focus on
headbands and electrode caps as these are non-invasive and require the least
expertise. These caps usually consists of small discs made of very conductive
material such as gold, silver or stainless steel. Commonly used electrodes are
made of Ag-AgCl disks with diameter of 1 to 3 mm and have long leads that can be
plugged into an amplifier. \cite{eegFund} The discs can be in direct contact
with scalp and measure electric potential produced by brain, but conductive
media in form of a gel or paste is often used to increase conductivity even more
by lowering contact impedance and improve readings at the lowest level, while it
also helps electrodes to stick to surface.

% Cap
The electrodes are pre-mounted on a silicon cap in locations according to
standardized placement systems that will be discussed later in this section.
This both speeds up the setup stage and allows for unified mapping of electrodes
to head surface, but fails to account for different shape and features of each
individual's head. \cite{eegFund} This problem is usually solved by involving a
method for locating 3-D coordinates of electrode positions on the head. Such
methods include using magnetic field digitizer or elaborate algorithms that can
calculate position of each electrode from few distance and angle measurements by
utilizing specific properties of standard placement system. \cite{rapidPos}

% Amplifiers

% Filters

% A/D converter

% Recording device

\subsubsection{Electrode placement systems}
% Uvod
The need for standardized electrode placement was evident as early as 1947 when
the first Internation EEG congress held place in London. Various method of
standardization were proposed and this effort finally resulted in the definition
of 10-20 electrode system in 1958 by H.H. Jasper. \cite{elSys} 
% Popis 10-20
This system consists of 21 electrodes placed evenly between certain landmarks.
(dokoncit)

% Popis 10-10
The 10-20 system offers only limited resolution but it was sufficient at the
time of its creation because the main limiting factor was technology. However,
with technological advances in computers and signal processing, using more than
21 channels became feasible and naturally many seeked to take advantage of
higher resolution. With more electrodes than what is 10-20 model capable of
accommodating, first extension to standard placement system was needed. The
extension was proposed in 1985 and extended the number of electrodes to 74. It
uses additional coronal contours lying halfway between original contours and
combines the labels of these original contours to create new label.
For example, the contour lying betweeen original countours F and C will be
labelled FC. \cite{elSys}

% Popis 10-5
The 10-5 electrode placement systems further extends the 10-10 system to allow
even higher number of channels to be used, as systems with over 128 channels are
not uncommon any more and have became commercially available. The idea in
placement of new electrodes is simmiliar to how 10-10 system extended from
10-20, creating coronal contours halfway between the original ones. The
nomenclature uses naming in similiar fashion as geographical directions. For
example, direction between North and North-West is labelled as North-North-West
and so contour between C-contour and CP-contour will be labelled CCP-contour.
This doubles the number of contours and also the number of electrodes in
contour, increasing the number of electrodes approximately 4 times. The total
number of possible locations is arround 345 and may vary depending on how many
electrodes are included for the most inferior rows. \cite{elSys} This system is
in proposition stage and is yet to be recognized as standard but it is currently
only work in this area, therefore can be considered as relevant.

\subsection{Measurment procedure}
Before measurment, the electodes are cleaned thoroughly to remove any impurities
that could impact measurement. The skin is also cleaned from oils and brushed
from dried parts. It is important to adhere to strict hygiene as
this could cause irritation and inflamation and with repeated measurements may
develop into an infection.\cite{eegFund} The subject is then seated comfortably
to eliminate any unnecessary movement, because it could spoil the measured data.
(cite?) Elecetrodes are placed on subject's scalp and recording can begin.
Subject is usually instructed to perform specific tasks (povenovat sa roznym
uloham co sa bezne vykonavaju).

\subsection{Data analysis}
Displaying and interpreting brainwave paterns at various frequencies -- alfa,
beta.. Identifying user's activity and what can be considered noise.
\chapter{Computer assistance}
\section{Brain-Computer interface} 
\section{Signal processing}
\emph{2 pages}
Introduction to sampling, fourier transform, various signal filters..
\section{Data visualization methods}
\emph{4 pages}
\subsection{2D representations}
graph, summary overview table, histogram, unfolded brain map..
\subsection{3D models}
mesh, volumetric/voxel graphics.. 
Means to obtain three-dimensional brain model - CT, modelling..
\section{Existing solutions}
\emph{2-3 solutions, 1 page each}
Analysis of existing solutions, what could serve as inspiration, what could be
improved from all points of view (ease of installation, usability, features
friendliness, signal processing, settings, extensibility..)
\subsection{A}
\subsection{B}
\subsection{C}
\chapter{Concept and design}
Theoretical concept for practical part.
\section{User interface}
\emph{2 pages}
Drawings, mockups, layouts, command prompt parameters.
\section{Code structure}
\emph{3 pages}
\subsection{Logical division}
The division on subsystems for graphics context setup, signal processing, gpu
rendering procedures, user interface and model importer.
\subsection{Classes and communication}
Structure of code defined by C++ classes, probably should be just global
overview with references to doxygen documentation, maybe some graphical code
maps.
\chapter{Implementation}
\emph{? pages}
Will cover implementation details, algorithms, significant mechanisms, shader
programming.. The logical division defined earlier will be probably used to
describe each part separately.
\section{Tools and libraries}
\emph{0.5-1 page}
Short enumeration, links or references to websites.
\section{Input data processor}
\subsection{Input format}
\subsection{Filters}
\section{Model importer}
\section{3D rendering system}
\section{Plugin system}
\section{Graphical user interface}
\section{\ldots}
\chapter{Results}
Summarization of results obtained by this work, final graphs and tables,
higlighting of the most interesting parts, presentation of user 
research -- opinions, suggestions,
feture requests..
\chapter{Discussion}
Discussing the results in regard to referenced literature and their results.
Showing significance of findings, questioning them, providing several
perspectives and means for argumentation.
\chapter{Conclusions}
Refering to introduction and checking with goals, if everything required was
done, what couldn't be done and why. Placing the work into wider context of
relevant areas of research -- medicine, bioinformatics, user interfaces..
Part about possible plans for future, improvment, enhancements, folow up.. \cite{Knuth}
%=========================================================================
 % viz. obsah.tex

  % Pouzita literatura
  % ----------------------------------------------
\ifczech
  \bibliographystyle{czechiso}
\else 
  \bibliographystyle{plain}
%  \bibliographystyle{alpha}
\fi
  \begin{flushleft}
  \bibliography{literatura} % viz. literatura.bib
  \end{flushleft}
  \appendix
  
  \chapter{Content of DVD}
This work contains a DVD with following structure:
\begin{itemize}
	\item \textbf{bin/} folder with a self extracting archive containing compiled application for Windows along with required libraries.
	\item \textbf{doc/} folder containing source code documentation generated by Doxygen. 
	\item \textbf{latex/} folder containing \LaTeX\ source codes for generating text of this thesis.
	\item \textbf{source/} folder containing source codes for building the application. Refer to the included \textbf{README.md} for instructions on how to build the application.
\end{itemize}
The content of the archive with compiled application is following:
\begin{itemize}
	\item \textbf{data/} folder containing EEG data in the EDF format for testing purposes.
	\item \textbf{electrodes/} folder which contains \texttt{default.elmap} specifying default electrode placement and \texttt{electrodes.obj} providing 3D electrode positions. 
	\item \textbf{licenses/} folder containing licenses for used libraries, EEG data and models.
	\item \textbf{models/} folder containing mesh models.
	\item \textbf{platforms/} folder with Qt libraries specific for Windows platform.
	\item \textbf{shaders/} folder with GLSL source codes that are compiled during run-time by the application.
	\item \textbf{bav.exe} -- the application executable.
	\item \textbf{.dll files} -- dynamic libraries required by the application.
	\item \textbf{MANUAL.txt} -- short manual describing basic interaction with the application.
\end{itemize}

\chapter{Image of Graphical User Interface}
\begin{figure}[htb]
	\centering
	\includegraphics[width=0.85\linewidth]{fig/gui.png}
\end{figure}

\chapter{Signal Filtering Tests}
The signal module which encompasses signal processing methods was tested against the MATLAB software to check the validity of implementation. First, a random signal with  $10000$ samples was generated. The signal was then filtered by the testing executable \textbf{sigTest.exe}. The testing executable is generated during build if the project was configured with \texttt{-DTEST\_PROJ=true}. The filtering was done in multiple configuration which are listed in table below.

\begin{table}[!h]
	\centering
	\label{t:filterCfg}
	\begin{tabular}{|c|c|c|c|c|}	
		\hline Filter & Window & Order & Cutoff frequency & Filename \\ 
		\hline Low-pass & Hamming & 10 & 5 & low5Hamm5 \\ 
		\hline Low-pass & Hamming & 10 & 10 & low10Hamm5 \\ 
		\hline Low-pass & Hamming & 20 & 5 & low5Hamm10 \\ 
		\hline Low-pass & Hamming & 20 & 10 & low10Hamm10 \\ 
		\hline Low-pass & Blackman & 10 & 5 & low5Black5 \\ 
		\hline Low-pass & Blackman & 10 & 10 & low10Black5 \\ 
		\hline Low-pass & Blackman & 20 & 5 & low5Black10 \\ 
		\hline Low-pass & Blackman & 20 & 10 & low10Black10 \\ 
		\hline High-pass & Hamming & 10 & 50 & high50Hamm5 \\ 
		\hline High-pass & Hamming & 10 & 100 & high100Hamm5 \\ 
		\hline High-pass & Hamming & 20 & 50 & high50Hamm10 \\ 
		\hline High-pass & Hamming & 20 & 100 & high100Hamm10 \\ 
		\hline High-pass & Blackman & 10 & 50 & high50Black5 \\ 
		\hline High-pass & Blackman & 10 & 100 & high100Black5 \\ 
		\hline High-pass & Blackman & 20 & 50 & high50Black10 \\ 
		\hline High-pass & Blackman & 20 & 100 & high100Black10 \\ 
		\hline 
	\end{tabular}
	\caption{Tested filter configurations}
\end{table}

Each configuration produced single output file with the filtered signal. This file was read by a MATLAB script and compared to the result of filtering using MATLAB filters in the same configuration. The maximum and average deviation was calculated and printed to output. The difference between signals filtered by testing executable and MATLAB in all configurations was marginal and can be attributed to floating point errors.

\chapter{Code metrics}
\vfill
\begin{table}[!h]
	\centering
	\label{t:codeMetrics}
	\begin{tabular}{|l|c|}
		\hline Files &  89 \\ 
		\hline Lines of code & 7517 \\
		\hline Percent of comments & 11.5\% \\ 
		\hline Statements & 3436 \\ 
		\hline Class definitions & 47 \\ 
		\hline Methods per class & 7.98 \\ 
		\hline Statements per method & 5.2 \\ 
		\hline Maximum depth & 6 \\
		\hline Average depth & 1.12 \\
		\hline 
	\end{tabular} 
	\caption{Code metrics}
\end{table}
\vfill
%\chapter{Manual}
 % viz. prilohy.tex
\end{document}
